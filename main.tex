\documentclass{article}
\usepackage[utf8]{inputenc}
\usepackage[spanish]{babel}
\usepackage{listings}
\usepackage{graphicx}
\graphicspath{ {images/} }
\usepackage{cite}

\begin{document}

\begin{titlepage}
    \begin{center}
        \vspace*{1cm}
            
        \Huge
        \textbf{PROYECTO INFORMATICA II}
            
        \vspace{0.5cm}
        \LARGE
        
            
        \vspace{1.5cm}
            
        \textbf{Esteban Felipe Güiza Piñeros}
            
        \vfill
            
        \vspace{0.8cm}
            
        \Large
        Despartamento de Ingeniería Electrónica y Telecomunicaciones\\
        Universidad de Antioquia\\
        Medellín\\
        Marzo de 2021
            
    \end{center}
\end{titlepage}

\section{Pasos para la Activida:}\dlabel

1. Buscar una Hoja de Papel (Tamaño Carta)


2. Encontrar 2 Tarjetas de iguales medidas.


3. Buscar una mesa con una   superficie estable y plana.


4.El estado inicial es que la hoja de papel esta sobre las tarjetas.

5.Ponga las 2 tarjetas sobre la hoja de papel.

6. Con una mano sobreponer ambas tarjetas de tal forma que todos sus bordes coincidan.

7. Con la misma mano llevar ambas tarjetas arrastrándolas contra la mesa hasta el borde de la hoja, sin que la hoja se mueva de su lugar y sin separar las tarjetas, dejando una pequeña parte de las tarjetas por fuera de la hoja.

8.Con la misma mano, utilizar los dedos pulgar e índice para tomar el borde de las tarjetas   que quedaron por fuera de la hoja, llevaras hacia arriba y retroceder un poco al centro de la hoja.

9. Mientras se sostiene las tarjetas con la misma mano colocarlas en forma vertical, poniendo el otro extremo contra la superficie de la hoja de papel, de tal forma que las tarjetas queden en un ángulo de 90 grados y el extremo que se sostiene con los dos dedos quede en la parte superior.

10. Con la misma mano deslizar la tarjeta que sostiene el pulgar hacia arriba, dejando una diferencia notable de altura entre ambas tarjetas.

11. La tarjeta que sostiene el pulgar debe ser un poco más alta que la que sostiene el dedo índice.

12. Después de lograr esa diferencia de altura, inclinar la tarjeta que sostiene el dedo índice hacia la otra tarjeta utilizando la superficie de la hoja y usando la otra tarjeta del pulgar como punto de apoyo.

13. Cuando la tarjeta del dedo incide tenga mucha inclinación con la misma mano ir inclinando la tarjeta que sostiene el dedo pulgar hacia el lado opuesto de una forma más leve, utilizar los dedos medio y anular para controlar el ángulo de inclinación y lograr que ambas logren equilibrarse y formen un triángulo con la base de la mesa. (Si el procedimiento falla, regresar al paso 5 e intentarlo de nuevo).


\section{Link del Video de los Pasos Registrados:}Link del Video: https://youtu.be/ymMaKhCEGjI
\end{document}
